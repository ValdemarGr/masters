\begin{abstract}
  When a programmer picks a language, the choices often falls into two schools, the imperative one and the functional one.
  The languages that fall into the category of imperative languages often deal with details of how to perform computation.
  However, functional programming languages deal with problems at higher abstraction levels, that are based in theoretic computer science.
  Functional programming languages have the benefit of being built on rigorous foundations, some of which guarantee correctness.

  In this work, we will demystify the workings of functional programming languages, beginning at definition and ending at evaluation.
  %We will explore the language of functional programming and the abstractions which can be built with and on top of them.
  We will explore the relationship between high level functional programming languages and the minimal lambda calculus.
  Thereafter, we will consider systems of verification to prove the correctness of programs through type inference and type checking.
  Once a program has been verified, efficient and minimal methods of practical evaluation of the language is considered, under various evaluation strategies.
  %In particular, we will explore the theory behind efficient functional programming languages and implement an efficient and minimal programming language as we progress through the chapters.
\end{abstract}
