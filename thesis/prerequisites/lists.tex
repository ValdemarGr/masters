
\section{Lists and immutability}
An instance of a data structure which has been thoroughly discussed throughout this thesis is \texttt{List} (\autoref{lst:listadt}).
\texttt{List} is an excellent choice as an introductory data structure since it gives insight into some very universal problems regarding both immutable and mutable data structures.
One is free to choose the operations for \texttt{List} but a common operation is \texttt{map} (\autoref{lst:map}).
\begin{lstlisting}[language=ML,caption={Mapping from \texttt{List a} to \texttt{List b}},label={lst:map},mathescape=true]
fun map l f = 
   match l
    | Cons x xs -> Cons (f x) (map xs f);
    | Nil -> l;
   ;
\end{lstlisting}
An interesting observation from map is that it runs differently in a call by need environment compared to a call by value environment.
In a call by value environment the operation takes $O(n)$ time since every \texttt{Cons}'ed value must be visited.
In a call by need environment the operation has its complexity reduced to $O(1)$ time since the result of the \texttt{map} operation becomes a suspended function, which will be evaluated when needed.


