\chapter{Conventional data structures and terminology}
Data structures in traditional contexts are \textit{homogeneous} collections of data, usually with a particular shape represented by an algebraic data type, with an associated set of \textit{morphisms}.
A homogeneous collection of data is a collection in which every element is of the same type.
Morphisms come in various forms, they essentially encapsulate the operations that can be performed on a data structure (or more generally an object).
Algebraic data structures and their associated morphisms come together into an algebra.
\begin{remark}
    In object oriented programming data structures (an algebra) is most often implemented through a class while functional programming languages often separate the shape and operations.
\end{remark}

Conventional data structures encapsulates data structures which are interesting under the call by value (\autoref{sec:es}) evaluation strategy.
Evaluation strategies have many implications on the data structure in question.
In call by name or call by need one would have to be careful not to create an unnecessary dependency which may force a computation which could otherwise stay suspended.
The choice of evaluation strategy and data structure implementation has a significant impact on complexity analysis, which will be explored.

\section{Lists and immutability}
An instance of a data structure which has been thoroughly discussed throughout this thesis is \texttt{List} (\autoref{lst:listadt}).
\texttt{List} is an excellent choice as an introductory data structure since it gives insight into some very universal problems regarding both immutable and mutable data structures.
One is free to choose the operations for \texttt{List} but some common ones include \texttt{filter} (\autoref{lst:filter}) and \texttt{map} (\autoref{lst:map}).
\begin{lstlisting}[float,language=ML,caption={Filtering some \texttt{l:List a} based on a predicate p},label={lst:filter},mathescape=true]
fun filter l p = 
    match l
        | Cons x xs -> 
            let tail = filter xs p;
            if (p x)
                Cons x tail;
            else
                tail;
            ;
        | Nil -> l;
    ;
\end{lstlisting}
\begin{lstlisting}[float,language=ML,caption={Mapping from \texttt{List a} to \texttt{List b}},label={lst:map},mathescape=true]
fun map l f = 
   match l
    | Cons x xs -> Cons (f x) (map xs f);
    | Nil -> l;
   ;
\end{lstlisting}
Under the condition of immutability there are several guarantees and 

