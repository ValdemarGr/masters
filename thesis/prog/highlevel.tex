\documentclass[11pt,oneside,a4paper]{report}

\begin{document}
\section{High level abstractions}\label{sec:highlevel}
The lambda calculus is a powerful language that can express any algorithm.
Expressiveness does not necessarily imply ergonomics or elegance, in fact encoding moderately complicated algorithms in lambda calculus becomes quite messy.
Many high level techniques exist to model abstractions in tangible concepts.

\subsection{Algebraic data types}
Algebraic data types are essentially a combination of disjoint unions, tuples and records.
Algebraic data types are closely related to types thus require some type theory to fully grasp.
Types are explored more in depth in \autoref{types}.
\begin{lstlisting}[language=ML,caption={List algebraic data type},label={lst:listadt}]
type List a = 
    | Nil
    | Cons a (List a)
;
\end{lstlisting}
\autoref{lst:listadt} is an implementation of a linked list. 
The list value can either take the type of \texttt{Nil} indicating an empty list, or it can take the type of \texttt{Cons} indicating a pair of type $a$ and another list.
The list implementation has two type constructors and one type parameter.
The type parameter $a$ of the list algebraic data type defines a \textit{polymorphic type}; $a$ can agree on any type, it is universally quantified \texttt{$\forall a. $ Cons $a$ (List $a$)}.
The two type constructors \texttt{Nil} and \texttt{Cons} both create a value of type \texttt{List a} once instatiated.
\begin{lstlisting}[language=ML,caption={List instance and match},label={lst:listinstance}]
let l = Cons 1 (Cons 2 (Cons 3 Nil));

match l 
    | Nil -> 0;
    | Cons n _ -> n;
;
\end{lstlisting}
Once a value is embedded into an algebraic data type such as a list it must be extractable to be of any use.
Values of algebraic data types are extracted and analysed with \textit{pattern matching}.
Pattern match comes in may forms, notably it allows one to define a computation based on the type an algebraic data type instance realizes (\autoref{lst:listinstance}).

\subsubsection{Scott encoding}
Pattern matching strays far from the simple untyped lambda calculus, but can in fact be encoded into it.
The \textit{scott encoding} (\autoref{se:general}) is a technique that describes a general purpose framework to encode algebraic data types into lambda calculus~\cite{scott1962system}.
Considering an algebraic data type instance as a function which accepts a set of ``handlers'' allows the encoding into lambda calculus.
The scott encoding specifies that type constructors should now be functions that are each parameterized by the type constructor parameters $x_1 \dots x_{A_i}$ where $A_i$ is the arity of the type constructor $i$.
Additionally each of the type constructor functions return a $n$ arity function, where $n$ is the cardinality of the set of type constructors.
Of the $n$ functions, the type constructor parameters $x_1 \dots x_{A_i}$ are applied to the $i$'th ``handler'' $c_i$.
These encoding rules ensure that the ``handler'' functions are provided uniformly to all instances of the algebraic data type.
\begin{align}
    \lambda x_1 \dots x_{A_i}. \lambda c_1 \dots c_n. c_i x_1 \dots x_{A_i}
\label{se:general}
\end{align}
\begin{exmp}
    The \texttt{List} algebraic data type in \autoref{lst:listadt} has two type constructors, \texttt{Nil} with the type constructor type \autoref{eq:nilconstructor} and \texttt{Cons} with the type constructor type \autoref{eq:consconstructor}.
    \autoref{eq:nilconstructor} is in fact also the type of \texttt{List} once instantiated, effetively treating partially applied functions as data.
\begin{align}
   &b \rightarrow (a \rightarrow \texttt{List } a \rightarrow b) \rightarrow b
   \label{eq:nilconstructor}\\
   &(a \rightarrow \texttt{List } a \rightarrow b) \rightarrow b \rightarrow (a \rightarrow \texttt{List } a \rightarrow b) \rightarrow b
   \label{eq:consconstructor}
\end{align}
\begin{lstlisting}[language=ML,caption={List algebraic data type implementation},label={lst:listscott}]
fun cons x xs = 
    fun c _ onCons = onCons x xs;
    c;

fun nil = 
    fun c onNil _ = onNil;
    c;
\end{lstlisting}
Encoding the constructors in $L$ yields the functions defined in \autoref{lst:listscott}.
Pattern matching is but a matter of applying the appropriate handlers.
In \autoref{lst:listscottexample}.
\begin{lstlisting}[language=ML,caption={Example of scott encoded list algebraic data type},label={lst:listscottexample}]
let l = cons 1 (cons 2 (cons 3 nil));

fun head x _ = 
    x;

l head 0;
\end{lstlisting}
\end{exmp}

Efficiency can be a bit tricky in lambda calculus as it is at the mercy of implementation.
A common method of considering efficiency is counting \textit{$\beta$-reduction} since they evaluate to function invocations.
The $\beta$-reduction is a substitution which substitutes an application where the left side is an abstraction in witch the bound variable is substituted with the right side term (\autoref{eq:betareduction}). 
\begin{align}
    &\beta_{red}((\lambda x . T) E) = T[x := E]
   \label{eq:betareduction}
\end{align}
It should be clear that invoking a $n$ arity function will take $n$ applications.
In the case of a scott encoded algebraic data types the largest term in regard to complexity is either the size of the set of ``handler'' functions or the ``handler'' function with most parameters.
The time to evaluate pattern match is thus $O(\max_i(c_i + A_i))$.



\end{document}
