\documentclass[11pt,oneside,a4paper]{report}

\begin{document}
\section{High level abstractions}\label{sec:highlevel}
The lambda calculus is a powerful language that can express any algorithm.
Expressiveness does not necessarily imply ergonomics or elegance, in fact encoding moderately complicated algorithms in lambda calculus becomes quite messy.

\subsection{Algebraic data types}
\label{sec:adt}
Algebraic data types, in their essence, are tagged unions of tuples.
Algebraic data types become much more powerful once enhanced with \textit{type constructors}, as such, they are closely related to types thus require some type theory to fully grasp.
Types are explored more in depth in \autoref{sec:types}.

An algebraic data type is a name $A$ for a tagged union of tuples $T_1, T_2 \dots , T_n$ that states that any tuple $T_k$ with tag $t_k$ for some $1 \leq k \leq n$ is of type $A$.
This construct allows any of the tuples $T_1, T_2 \dots , T_n$ to be unified and inhabit $A$ under the names $t_1, t_2 \dots , t_n$.

In $L$, algebraic data types give rise to tangible implementations of abstract types such as lists (\autoref{lst:listadt:int}).
\autoref{lst:listadt:int} displays an algebraic data type with name \texttt{IntList} which is the tagged union of the nullary tuple \texttt{Nil} and the binary tuple \texttt{Cons}.
\texttt{IntList} states that if a value that inhabits the type \texttt{IntList} occurs, it must either be the tag \texttt{Nil} or the tag \texttt{Cons} which carries a value of type \texttt{Int} and another value of type \texttt{IntList}.
\begin{figure}
\begin{lstlisting}[language=ML,caption={List algebraic data type},label={lst:listadt:int}]
type IntList = 
    | Nil
    | Cons Int IntList
;
\end{lstlisting}
\end{figure}
%\begin{figure}
%\begin{lstlisting}[language=ML,caption={List algebraic data type},label={lst:listadt}]
%type List a = 
    %| Nil
    %| Cons a (List a)
%;
%\end{lstlisting}
%\end{figure}
%%\autoref{lst:listadt} is an implementation of a linked list. 
%The list value can either take the type of \texttt{Nil} indicating an empty list, or it can take the type of \texttt{Cons} indicating a pair of type $a$ and another list.
%The list implementation has two constructors and one type parameter.
%The type parameter $a$ of the list algebraic data type defines a \textit{polymorphic type}; $a$ can agree on any type, it is universally quantified \texttt{$\forall a. $ Cons $a$ (List $a$)}.
%The two constructors \texttt{Nil} and \texttt{Cons} both create a value of type \texttt{List a} once instatiated.
\begin{figure}
\begin{lstlisting}[language=ML,caption={List instance and match},label={lst:listinstance}]
fun main =
  let l = Cons 1 (Cons 2 (Cons 3 Nil));
  match l 
      | Nil -> 0;
      | Cons x xs -> x;
  ;
\end{lstlisting}
\end{figure}
Once a tuple of values are embedded by a tag into an algebraic data type, such as a list, it must be extractable, to be of any use.
Values of algebraic data types are extracted and analysed with \textit{pattern matching}.
Pattern matching comes in many forms, notably it may allow one to define a computation based on the type an algebraic data type instance realizes (\autoref{lst:listinstance}), which is the type of pattern matching of $L$.

\subsubsection{Scott encoding}
Pattern matching strays far from the simple untyped lambda calculus, but can in fact be encoded into it.
\textit{scott encoding} (\autoref{se:general}) is a technique that describes a general purpose framework to encode algebraic data types into the lambda calculus~\cite{scott1962system}.
Considering an algebraic data type instance as a function which accepts a set of ``handlers'', paves the way for the encoding into the lambda calculus.
The scott encoding specifies that constructors should now be functions that are each parameterized by the constructor parameters $x_1 \dots x_{A_i}$ where $A_i$ is the arity of the constructor $i$.
Additionally, each of the constructor functions return a $n$ arity function, where $n$ is the number of tagged tuples $T_1, T_2 \dots , T_n$.
Of the $n$ functions, the constructor parameters $x_1 \dots x_{A_i}$ are applied to the $i$'th ``handler'' $c_i$.
These encoding rules ensure that the ``handler'' functions are provided uniformly to all instances of the algebraic data type.
\begin{align}
    \lambda x_1 \dots x_{A_i}. \lambda c_1 \dots c_n. c_i x_1 \dots x_{A_i}
\label{se:general}
\end{align}
\begin{exmp}
    The \texttt{IntList} algebraic data type in \autoref{lst:listadt:int} has two constructors, the \texttt{Nil} constructor and the \texttt{Cons} constructor.
    The construction of a value of type \texttt{Cons} or \texttt{Nil} effectively partially applies an abstraction and returns an abstraction that is uniform for both \texttt{Nil} and \texttt{Cons}, such as in (\autoref{lst:listscott}).
    %\autoref{eq:nilconstructor} is in fact also the type of \texttt{List} once instantiated, effetively treating partially applied functions as data.
\begin{figure}
\begin{lstlisting}[language=ML,caption={List algebraic data type implementation},label={lst:listscott}]
fun cons x xs = 
    fun c onNil onCons = onCons x xs;
    c;

fun nil = 
    fun c onNil onCons = onNil;
    c;
\end{lstlisting}
\end{figure}

Types have not been introduced yet, but seeing the types of these functions might help understanding scott encoding.
\autoref{eq:nilconstructor} is the constructor type for \texttt{Nil} and \autoref{eq:consconstructor} is the constructor type for \texttt{Cons}.
\begin{align}
   &b \rightarrow (a \rightarrow \texttt{List } a \rightarrow b) \rightarrow b
   \label{eq:nilconstructor}\\
   &(a \rightarrow \texttt{List } a \rightarrow b) \rightarrow b \rightarrow (a \rightarrow \texttt{List } a \rightarrow b) \rightarrow b
   \label{eq:consconstructor}
\end{align}

Encoding the constructors in $L$ yields the functions defined in \autoref{lst:listscott}.
Pattern matching is but a matter of applying the appropriate handlers.
In \autoref{lst:listscottexample}.
\begin{lstlisting}[language=ML,caption={Example of scott encoded list algebraic data type},label={lst:listscottexample}]
fun main =
  let l = cons 1 (cons 2 (cons 3 nil));
  fun consCase x xs = x;
  fun nilCase = 0;
  l consCase nilCase
\end{lstlisting}
\end{exmp}

%Efficiency can be a bit tricky in lambda calculus as it is at the mercy of implementation.
%A common method of considering efficiency is counting \textit{$\beta$-reduction} since they evaluate to function invocations.
%The $\beta$-reduction is a substitution which substitutes an application where the left side is an abstraction in witch the bound variable is substituted with the right side term (\autoref{eq:betareduction}). 
%\begin{align}
    %&\beta_{red}((\lambda x . T) E) = T[x := E]
   %\label{eq:betareduction}
%\end{align}
%It should be clear that invoking a $n$ arity function will take $n$ applications.
%In the case of a scott encoded algebraic data types the largest term in regard to complexity is either the size of the set of ``handler'' functions or the ``handler'' function with most parameters.
%The time to evaluate pattern match is thus $O(\max_i(c_i + A_i))$.



\end{document}
