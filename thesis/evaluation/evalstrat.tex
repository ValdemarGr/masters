\section{Evaluation strategies}
\label{sec:es}
When evaluating the untyped lambda calculus one has to choose an evaluation strategy.
The choice of evaluation strategy has a large impact on aspects such as complexity guarantees.
Such strategies are \textit{call by value}, \textit{call by name} and \textit{call by need}.
Call by value is most often the simplest and most natural way of assuming program execution.
\begin{lstlisting}[language=ML,caption={Program that doubles values},label={lst:callbyvalue},mathescape=true]
fun main =
  fun double x = x + x;
  let a = double 10;
  double 10;
\end{lstlisting}
By the call by value semantics, \autoref{lst:callbyvalue} eagerly evaluates every expression.
Clearly the variable \texttt{a} is never used but under the call by value semantics everything is eagerly evaluated.
Every expression is evaluated in logical order in the call by value evaluation strategy.
\newline
\begin{minipage}{\textwidth}
\begin{lstlisting}[language=ML,caption={Implementation of call by name},label={lst:callbyname},mathescape=true]
fun main = 
  fun suspend x unit = x;
  fun force x = x 0;
  let value = suspend 10;
  fun double x = 
      fun susExpensiveOp unit = 
          (force x) + (force x);
      susExpensiveOp;
  let a = double value;
  force (double value);
\end{lstlisting}
\end{minipage}
The call by name semantics however does only evaluate expressions once they are needed.
By the call by name semantics \texttt{a} is never evaluated since it is never used.
In \autoref{lst:callbyname} call by name has been implemented by the use of various functions such as the two constant functions \texttt{suspend} and \texttt{force}.
\texttt{susExpensiveOp} ensures that the forcing (evaluation) of \texttt{x} never occurs until the caller of \texttt{double} forces the result.
By the aforementioned semantics of call by name in the context of the program in \autoref{lst:callbyname} \texttt{a} is never forced thus the computation is never performed.
The implementation of call by name can become quite troublesome and therefore in most cases is a part of the native execution environment which will be discussed in \autoref{tbd}.

The call by need strategy introduces \textit{lazy evaluation} semantics which is the same as call by name with one extra detail named \textit{sharing}.
In \autoref{lst:callbyname} \texttt{force x} is computed twice which may be an expensive operation.
Under call by need all results are saved for later use similar to techniques such as dynamic programming.
\begin{figure}
    \centering
    \begin{subfigure}[b]{0.33\textwidth}
        \centering
        \begin{tikzpicture}[ scale=0.8, every node/.style={scale=0.8}, node distance = 0.3cm and 0.3cm]
                \node[circle, draw=black] (app1) {\texttt{app}};
                    \node[circle, draw=black, below right = of app1] (force) {\texttt{force}};
                    \path[-] (app1) edge node[left] {} (force);

                    \node[circle, draw=black, below left = of app1] (app2) {\texttt{app}};
                    \path[-] (app1) edge node[left] {} (app2);
                        \node[circle, draw=black, below left = of app2] (value) {\texttt{value}};
                        \path[-] (app2) edge node[left] {} (value);
                        \node[circle, draw=black, below right = of app2] (double) {\texttt{double}};
                        \path[-] (app2) edge node[left] {} (double);

                \node[draw,scale=1.3,fill=none,rectangle,fit=(app1)(value)(force)](fstB){};
        \end{tikzpicture}
        \caption{The last expression of the program.}
        \label{sub:eval:main}
    \end{subfigure}
    \begin{subfigure}[b]{0.66\textwidth}
        \centering
        \begin{tikzpicture}[ scale=0.8, every node/.style={scale=0.8}, node distance = 0.3cm and 0.3cm]
                \node[circle, draw=black] (lamx) {\texttt{$\lambda$x}};
                    \node[circle, draw=black, below = of lamx] (lamu) {\texttt{$\lambda$unit}};
                    \path[-] (lamx) edge node[left] {} (lamu);
                        \node[circle, draw=black, below = of lamu] (app1) {\texttt{app}};
                        \path[-] (lamu) edge node[left] {} (app1);
                            \node[circle, draw=red, below right = of app1] (app2) {\texttt{app}};
                            \path[-] (app1) edge node[left] {} (app2);
                                \node[circle, draw=red, below right = of app2] (x1) {\texttt{x}};
                                \path[-] (app2) edge node[left] {} (x1);
                                \node[circle, draw=red, below = of app2] (force1) {\texttt{force}};
                                \path[-] (app2) edge node[left] {} (force1);

                            \node[circle, draw=black, below left = of app1] (app4) {\texttt{app}};
                            \path[-] (app1) edge node[left] {} (app4);
                                \node[circle, draw=black, below left = of app4] (add) {\texttt{+}};
                                \path[-] (app4) edge node[left] {} (add);

                                \node[circle, draw=red, below right = of app4] (app3) {\texttt{app}};
                                \path[-] (app4) edge node[left] {} (app3);
                                    \node[circle, draw=red, below = of app3] (x2) {\texttt{x}};
                                    \path[-] (app3) edge node[left] {} (x2);
                                    \node[circle, draw=red, below left = of app3] (force2) {\texttt{force}};
                                    \path[-] (app3) edge node[left] {} (force2);
                \node[draw,scale=1.3,fill=none,rectangle,fit=(lamx)(force2)(x1)(add)](fstB){};
        \end{tikzpicture}
    %\begin{tikzpicture}
        %\node[circle, draw=black] (lamx) {$\lambda \texttt{x}$};

        %\node[circle, draw=black, below = of lamx] (lamunit) {$\lambda \texttt{unit}$};

        %\node[circle, draw=black, below = of lamunit] (addition) {$+$};

        %\node[circle, draw=black, below left = of addition] (force1) {\texttt{force}};
        %\node[circle, draw=black, below = of force1] (x1) {\texttt{x}};
        
        %\node[circle, draw=black, below right = of addition] (force2) {\texttt{force}};
        %\node[circle, draw=black, below = of force2] (x2) {\texttt{x}};

        %\path[->] (lamx) edge node[left] {} (lamunit);
        %\path[->] (lamunit) edge node[left] {} (addition);

        %\path[->] (addition) edge node[left] {} (force1);
        %\path[->] (addition) edge node[left] {} (force2);

        %\path[->] (force1) edge node[left] {} (x1);
        %\path[->] (force2) edge node[left] {} (x2);
    %\end{tikzpicture}
        \caption{The expression tree for \texttt{double}}
        \label{sub:eval:double}
    \end{subfigure}
    \caption{}
    \label{fig:evalexpr}
\end{figure}
To understand this better observe the expression tree for \autoref{lst:callbyname} in \autoref{fig:evalexpr}.
Clearly the two red subtrees in \autoref{sub:eval:double} are identical thus they may be memoized such that the forcing of \texttt{x} only occurs once.
More generally if the execution environment supports lazy evaluation, once an expression has been forced it is remembered.

\section{Runtime environments}
Now that the untyped lambda calculus has been introduced, implemented and validated efficiently the question of execution naturally follows.
There exists many different well understood strategies to implement an execution environment for the untyped lambda calculus.
Naively it may seem straightforward to evaluate the untyped lambda calculus mechanically by $\beta$-reductions, but doing so brings upon some problems when implementing an interpreter.
%When applying the term $f x$ in $\lambda f . \lambda x . f x$ such that $f[? := x]$ where $?$ is the parameter name of $f$ it should become clear why a naive strategy is not enough since the parameters of $f$ are anonymous.

