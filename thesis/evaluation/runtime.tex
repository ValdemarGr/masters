\documentclass[11pt,oneside,a4paper]{report}

\begin{document}

\section{Evaluation strategies}
When evaluating the untyped lambda calculus one has to choose an evaluation strategy.
The choice of evaluation strategy has a large impact on aspects such as complexity guarantees.
The names of such strategies are \textit{call by value}, \textit{call by name} and \textit{call by need}.
The call by value is most often the simplest and most natural way of assuming program execution.
\begin{lstlisting}[language=ML,caption={Program that doubles values},label={lst:callbyvalue},mathescape=true]
fun double x = x + x;
let a = double 10;
double (double 10);
\end{lstlisting}
By the call by value semantics, \autoref{lst:callbyvalue} eagerly evaluates every expression.
Clearly the variable \texttt{a} is never used but under the call by value semantics everything is eagerly evaluated.
Every expression is evaluated in logical order in the call by value evaluation strategy.
\begin{figure}[ht]
\begin{lstlisting}[language=ML,caption={Implementation of call by name},label={lst:callbyname},mathescape=true]
fun suspend x unit = x;
fun force x = x 0;
let value = suspend 10;
fun double x = 
    fun susExpensiveOp unit = 
        (force x) + (force x);
    susExpensiveOp;
let a = double value;
force (double value);
\end{lstlisting}
\end{figure}
The call by name semantics however does only evaluate expressions once they are needed (also commonly called \textit{lazy evaluation}).
By the call by name semantics \texttt{a} is never evaluated since it is never used.
In \autoref{lst:callbyname} call by name has been implemented by the use of various functions such as the two constant functions \texttt{suspend} and \texttt{force}.
\texttt{susExpensiveOp} ensures that the forcing (evaluation) of \texttt{x} never occurs until the caller of \texttt{double} forces the result.
By the aforementioned semantics of call by name in the context of the program in \autoref{lst:callbyname} \texttt{a} is never forced thus the computation is never performed.
The implementation of call by name can become quite troublesome and therefore in most cases is a part of the native execution environment which will be discussed in \autoref{tbd}.

The call by need semantics introduces the same lazy evaluation semantics as the call by name strategy but with one extra detail named \textit{sharing}.
In \autoref{lst:callbyname} \texttt{force x} is performed twice which may be an expensive operation.
Under call by need all non side-effectful operations' results are saved for later use like techniques such as dynamic programming.
\begin{figure}
    \centering
    \begin{subfigure}[b]{0.33\textwidth}
        \centering
    \begin{tikzpicture}
        \node[circle, draw=black] (force) {\texttt{force}};

        \node[circle, draw=black, below = of force] (double) {\texttt{double}};

        \node[circle, draw=black, below = of double] (value) {\texttt{value}};

        \path[->] (force) edge node[left] {} (double);
        \path[->] (double) edge node[left] {} (value);
    \end{tikzpicture}
        \caption{The last expression of the program.}
        \label{sub:eval:main}
    \end{subfigure}
    \begin{subfigure}[b]{0.66\textwidth}
        \centering
    \begin{tikzpicture}
        \node[circle, draw=black] (lamx) {$\lambda \texttt{x}$};

        \node[circle, draw=black, below = of lamx] (lamunit) {$\lambda \texttt{unit}$};

        \node[circle, draw=black, below = of lamunit] (addition) {$+$};

        \node[circle, draw=black, below left = of addition] (force1) {\texttt{force}};
        \node[circle, draw=black, below = of force1] (x1) {\texttt{x}};
        
        \node[circle, draw=black, below right = of addition] (force2) {\texttt{force}};
        \node[circle, draw=black, below = of force2] (x2) {\texttt{x}};

        \path[->] (lamx) edge node[left] {} (lamunit);
        \path[->] (lamunit) edge node[left] {} (addition);

        \path[->] (addition) edge node[left] {} (force1);
        \path[->] (addition) edge node[left] {} (force2);

        \path[->] (force1) edge node[left] {} (x1);
        \path[->] (force2) edge node[left] {} (x2);
    \end{tikzpicture}
        \caption{The expression tree for \texttt{double}}
        \label{sub:eval:double}
    \end{subfigure}
    \caption{}
    \label{fig:evalexpr}
\end{figure}
To understand this better observe the expression tree for \autoref{lst:callbyname} in \autoref{fig:evalexpr}.
Clearly the right and left branches in \autoref{sub:eval:double} are identical thus they may be memoized such that the forcing of \texttt{x} only occurs once.
More generally if the execution environment supports lazy evaluation, once an expression has been forced it is remembered if that branch is executed again.

\end{document}
