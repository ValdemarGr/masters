\section{Combinator reducers}
\label{sec:comb}
One of the most prominent techniques for evaluating functional programs is that of \textit{combinator graphs reductions}.
Formally a combinator is a function that has no free variables which is convenient since the problem of figuring out closures and parameter substitutions in applications never arises.
\begin{align}
    x \label{eq:comb:x}\\
    F \label{eq:comb:F}\\
    Y E \label{eq:comb:app}
\end{align}
There are three types of terms in combinator logic; the variable much like the lambda calculus (\autoref{eq:comb:x}), application (\autoref{eq:comb:app}) and the combinator (\autoref{eq:comb:F}).
The SKI calculus is a very simple set of combinators which are powerful enough to be turing complete and translate to and from the lambda calculus.
In SKI $F ::= S \,\,|\,\, K \,\,|\,\, I$ where the equivalent lambda calculus combinators for $S = \lambda x . \lambda y . \lambda z . x z (y z)$, $K = \lambda x . \lambda y . x$ and $I = \lambda x . x$.
Evaluating an SKI program is a straightforward reduction where $F'_F$ denotes combinator $F'$ has been partially applied with combinator $F$.
\begin{exmp}
    \begin{align}
        &SKSI\\
        = &KI(SI)\tag*{}\\
        = &K_I(SI)\tag*{}\\
        = &I\tag*{}
    \end{align}
\end{exmp}

The algorithm for converting a lambda calculus program into a SKI combinator program is a straightforward mechanical one.
The evaluation context is always an abstraction $\lambda x . E$.
\begin{pcases}
    \pcase{$E = x$ then rewrite $\lambda x . E$ to $I$.}
    \pcase{$E = y$ where $y \neq x$ and $y$ is a variable then rewrite $\lambda x . y$ to $Ky$.}
    \pcase{$E = Y E'$ then rewrite $\lambda x . Y E'$ to $S(\lambda x . Y)(\lambda x . E')$ since applying some $y$ to $\lambda x . Y E'$ must lambda lift $y$ as a parameter named $x$ to both $Y$ and $E'$ such that the lifted expression becomes $((\lambda x . Y)y)((\lambda x . E')y) = S(\lambda x . Y)(\lambda x . E')y$.
    Then recurse in both branches.
    }
    \pcase{$E = \lambda x . E'$ then first rewrite $E'$ with the appropriate cases recursively such that $E'$ becomes either $x$, $y$ or $Y E$ such that Case 1, 2 or 3 can be applied.}
\end{pcases}
The termination of the rewriting to SKI is guaranteed since abstractions are always eliminated and the algorithm never introduce any additional abstractions.
When translating the untyped lambda calculus to SKI the ''magic`` variable names $\sigma, \kappa$ and $\iota$ are used as placeholder functions for the SKI combinators since the translation requires a lambda calculus form.
When the translation has been completed then replace $\sigma \mapsto S, \kappa \mapsto K, \iota \mapsto I$.
\subsection{Combinator translation growth}
Before proving that the SKI translation algorithm produces a program of larger size the notion of size must be established.
Size in terms of lambda calculus are the number of lambda terms (\autoref{lc:lang:abs}, \autoref{lc:lang:var} and \autoref{lc:lang:app}) that make up a program.
For instance $\lambda x . x$ has a size of two since it is composed of an abstraction and a variable term.
The size of an SKI combinator program is in terms of the number of combinators.
\begin{lemma}
    There exists a family of lambda calculus programs of size $n$ which are translated into SKI-expressions of size $\Omega(n^2)$.
\end{lemma}
\begin{proof}
\begin{pcases}
    \pcase{\label{eval:case:1} Rewriting $\lambda x . x$ to $I$ is a reduction of one.}
    \pcase{\label{eval:case:2} Rewriting $\lambda x . y$ to $Ky$ is equivalent in terms of size.}
    \pcase{Rewriting $\lambda x . YE$ to $S(\lambda x . Y)(\lambda x . E)$ is the interesting case.
        To induce the worst case size \autoref{eval:case:1} must be avoided.
        If $x \notin \textit{Free}(Y)$ and $x \notin \textit{Free}(E)$ then for every non-recursive term in $Y$ and $E$ \autoref{eval:case:2} is the only applicable rewrite rule which means that an at least equal size is guaranteed.
        Furthermore observe that by introducing unused parameters one can add one $K$ term to \textit{every} non-recursive case.
        Observe the instance $\lambda f_1 . \lambda f_2 . \lambda f_3 . (f_1 f_1 f_1)$ where the two unused parameters are used to add $K$ terms to all non-recursive cases in \autoref{eq:eval:red2} such that the amount of extra K terms minus the I becomes $\text{variable\_references} * (\text{unused\_abstractions} - 1) = 3*(3-1)$:
        \begin{gather}
            S(S(KKI)(KKI))(KKI)\label{eq:eval:red2}
        \end{gather}
        Now let the number of variable references be $n$ and the unused abstractions also be $n$ clearly $\Omega(n * (n - 1)) = \Omega(n^2)$
    }
    \pcase{Rewriting $\lambda x . E'$ is not a translation rule so the cost is based on what $E'$ becomes.}
\end{pcases}
    \vspace*{0.5cm}
Notice that the applications $f_1 f_1 \dots f_1$ can in fact be changed to $f_1 f_2 \dots f_n$ since for every $f_k$ where $0 < k \leq n$ there are $n - 1$ parameters that induce a K combinator.
Let $P_n$ be family of programs with $n$ abstractions and $n$ applications.
$\lambda f_1 . \lambda f_2 . \lambda f_3 . (f_1 f_1 f_1) \in P_3$ and in fact for any $p$ where $\forall n \in \mathbb{Z}^+$ and $p \in P_n$, $p$ translates into SKI-expressions of size $\Omega(n^2)$.
\end{proof}
\begin{exmp}
    Observe the size of \autoref{eq:eval:comp1} in comparison to \autoref{eq:evaltime}.
\begin{align}
    &\lambda f_1 . \lambda f_2 . f_1 f_2 \label{eq:eval:comp1}\\
    =&\lambda f_1 . \sigma(\lambda f_2 . f_1)(\lambda f_2 . f_2) \tag*{} \\
    =&\lambda f_1 . (\sigma(\kappa f_1))(\iota) \tag*{} \\
    =&\sigma (\lambda f_1 . \sigma (\kappa f_1)) (\lambda f_1 . \iota) \tag*{} \\
    =&\sigma (\sigma (\lambda f_1 . \sigma) (\lambda f_1 . \kappa f_1)) (\kappa \iota) \tag*{} \\
    =&\sigma (\sigma (\kappa \sigma) (\sigma (\lambda f_1 . \kappa) (\lambda f_1 . f_1))) (\kappa \iota) \tag*{} \\
    =&\sigma (\sigma (\kappa \sigma) (\sigma (\kappa \kappa) (\iota))) (\kappa \iota) \tag*{} \\
    =&S (S (K S) (S (K K) (I))) (K I) \tag*{}
\end{align}
\end{exmp}
It should become clear that many programs suffer from this consequence such as \texttt{let add = ($\lambda$x.$\lambda$y.(+ x) y)} $\in P_2$ where the program is written in prefix notation.
Translating the lambda calculus into the SKI-expressions does indeed increase the size significantly but does not warrant a write off entirely.
More advanced techniques exist to translate the lambda calculus to linearly sized SKI-expressions with the introduction of more complicated combinators~\cite{kiselyov2018lambda}.
