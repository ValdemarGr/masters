\chapter{Introduction and preliminaries}
At the time that this thesis was created, imperative programming languages are by far the most dominant in development of software.
Imperative languages are the natural materialization of how we act, since they in their essence are encoded in a way similar to how we behave.

\section{A case for immutability}
Making a sandwich, for instance, involves following a recipe, or in computational terms an algorithm.
\begin{figure}[ht]
  \begin{enumerate}
    \item Heat two slices of bread.
    \item Butter the heated bread.
    \item Boil eggs.\label{boil}
    \item Slice tomatoes\label{tomatoes}.
    \item Dice cucumber.\label{dice}
    \item Slice eggs.
    \item Place tomatoes on the bottom slice of the buttered bread bread.
    \item Place the diced cucumber on top of the tomatoes.
    \item Place sliced eggs on top of the diced cucumber.
    \item Place the top slice of the buttered bread on the eggs. 
  \end{enumerate}
  \caption{A recipe for sandwiches}
  \label{sandwich}
\end{figure}
An imperative algorithm for \autoref{sandwich} is very much alike the recipe in that, it defines the steps required to make a sandwich.
In a traditional imperative programming language, such as C or Java, the sandwich algorithm would involve changing or \textit{mutating} the sandwich at every step.
Mutation is the natural way of applying changes to an object, which is where functional programming languages very often differ.
\begin{remark}
When functional programming languages are mentioned, it should now be assumed that they are immutable.
\end{remark}

In immutable functional programming languages one cannot mutate an object, but rather create a new one with the desired changes.
The naive intuition of this change might indicate that immutability and mutability are opposites, which is necessarily not the case.
Let \texttt{heatBread} be a function that creates a heated slice of bread and let \texttt{butter} be a function that creates a copy of a heated slice of bread, and returns it buttered.
\begin{figure}
\centering 
  \begin{tikzpicture}
                \node[draw=black] (s1) {\texttt{Heat bread}};
                \node[draw=black, right = 2.5cm of s1] (s2) {\texttt{Butter the heated bread}};
                \path[->] (s1) edge node[left] {} (s2);

                \node[draw=black, below = of s1] (p1) {\texttt{heated = heatBread}};
                \node[draw=black, below = of s2] (p2) {\texttt{buttered = butter(heated)}};
                \path[->] (p1) edge node[left] {} (p2);

                \path[->] (s1) edge node[left] {} (p1);
                \path[->] (s2) edge node[left] {} (p2);
  \end{tikzpicture}
  \caption{An abstract and concrete representation of bread}
\end{figure}
The abstraction of immutability might not seem unnatural, if we imagine that every step is performed at a particular point in time.
Immutability introduces the abstraction of travelling in time, by never changing objects.
In this context, mutability implies the ability to travel back in time and change an action.
If one performs a seres of immutable steps in the sandwich algorithm and fails at say, \autoref{tomatoes}, then one can simply travel back in time and begin from the previous step.

This is the primary philosophy behind immutability, and has other practical benefits that from from the guarantee of immutable data.
For instance, in an attempt to introduce more people to help create a sandwich, one might run into problems when multiple people would want to change the sandwich.
In an immutable setting, the sandwich cannot be changed, thus everyone gets to pluck the old sandwich out of time and perform their job.

Throughout this work, all algorithms and data structures will be used presented and used immutably.
%If we had, say two more people, helping creating a sandwich, the algorithm in the context of imperative programming is quite different.
%Clearly, steps such as \autoref{boil}, \autoref{tomatoes} and \autoref{dice} are not mutually dependent, thus can be performed at the same time e.g. they can be performed in parallel.
%In an imperative mutable context, once one step is done, the sandwich must be acquired by one cook only, since changing the sandwich at the same time might lead to various.

%The if one heats a slice of bread at some point nn time, it hap
%One could argue that immutability introduces the notion of time to a program.

\section{Purity of programs}
A programming language where all variables can be replaced by their expressions is called \textit{pure}.
Purity implies that there can be state, mutability and side effect, since all of these concepts can violate the deterministic nature of purity.
Purity is a powerful property, since it allows us to reason about our whole program at once. 
In fact, if a program is pure, any number of substitutions and rewritings can be performed on the program without altering the meaning.
The concept of purity becomes very helpful once we begin to consider evaluation.

\section{Preliminaries}

\subsection{Notation}
Some functions are written in pattern matching style.
Functions which have parameters that can take different values or shapes, can implement a case for each value or shape.
For instance, a function which finds the cardinality of a set can be implemented as in \autoref{eq:notation:card}.
\begin{align}
  \texttt{card($\emptyset$) }&\texttt{= 0} \label{eq:notation:card} \\
  \texttt{card(\{x\} $\cup$ S) }&\texttt{= 1 + card(S)} \tag*{}
\end{align}
Inevitably, to allow complicated functions, like algorithms, some functions may require subexpressions.
Subexpressions are denoted with \texttt{where} when some expression is a composite of multiple expressions like in \autoref{eq:notation:fib}.
\begin{align}
  \texttt{fib(0) }&\texttt{= 0} \label{eq:notation:fib} \\
  \texttt{fib(1) }&\texttt{= 1} \tag*{} \\
  \texttt{fib(n) }&\texttt{= F$_\texttt{n-1}$ + F$_\texttt{n-2}$} \tag*{} \\
  \wherecase{F$_\texttt{n-1}$ = fib(n - 1)},\tag*{}\\
  \extracase{F$_\texttt{n-2}$ = fib(n - 2)}\tag*{}
\end{align}
Functions can also contain nested functions such as in \autoref{eq:nested}.
\begin{align}
  \texttt{double(n) }&\texttt{= k + id(n)} \label{eq:nested} \\
  \wherecase{id(x) = x,} \tag*{}\\
  \extracase{k = id(n)} \tag*{}
\end{align}
Finally, conditional cases may also occur in functions such that the choice of action is dependent on a predicate.
The function in \autoref{eq:cases} increments \texttt{n} in the case that it is even, else it returns \texttt{n}.
\begin{align}
  \texttt{makeOdd(n) }&\texttt{=} 
  \begin{cases}
    \texttt{n + 1} & \texttt{ if n \texttt{mod} 2 = 0 }\\
    \texttt{n} & 
  \end{cases}\label{eq:cases} 
\end{align}


\subsection{Running and sources}
The compiler and interpreter for the programming language implemented in this thesis was written in Scala.
To compile the compiler and interpreter, one must install sbt (simple build tool) and bloop (build server).
To run a compile and run a program the program bloop must be invoked with the arguments \texttt{bloop run root -- /path/to/code/src/L/fib}, where the absolute path is the path to the file containing the code for the implemented programming language.
Every resource required for this thesis can be found in the git repository at \url{https://github.com/valdemargr/masters}.

