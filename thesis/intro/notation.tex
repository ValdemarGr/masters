\section{Notation}
Some functions are written in pattern matching style.
Functions which have parameters that can take different values or shapes can implement a case for each value or shape.
For instance, a function which finds the cardinality of a set can be implemented as in \autoref{eq:notation:card}.
\begin{align}
  \texttt{card($\emptyset$) }&\texttt{= 0} \label{eq:notation:card} \\
  \texttt{card(\{x\} $\cup$ S) }&\texttt{= 1 + card(S)} \tag*{}
\end{align}
Inevitably, to allow complicated functions, like algorithms, some functions may require subexpressions.
Subexpressions are denoted with \texttt{where} when some expression is a composite of multiple expressions like in \autoref{eq:notation:fib}.
\begin{align}
  \texttt{fib(0) }&\texttt{= 0} \label{eq:notation:fib} \\
  \texttt{fib(1) }&\texttt{= 1} \tag*{} \\
  \texttt{fib(n) }&\texttt{= F$_\texttt{n-1}$ + F$_\texttt{n-2}$} \tag*{} \\
  \wherecase{F$_\texttt{n-1}$ = fib(n - 1)},\tag*{}\\
  \extracase{F$_\texttt{n-2}$ = fib(n - 2)}\tag*{}
\end{align}
Functions can also contain nested functions such as in \autoref{eq:nested}.
\begin{align}
  \texttt{double(n) }&\texttt{= k + id(n)} \label{eq:nested} \\
  \wherecase{id(x) = x,} \tag*{}\\
  \extracase{k = id(n)} \tag*{}
\end{align}
Finally, conditional cases may also occur in functions such that the choice of action is dependent on a predicate.
The function in \autoref{eq:cases} increments \texttt{n} in the case that it is even, else it returns it.
\begin{align}
  \texttt{makeOdd(n) }&\texttt{=} 
  \begin{cases}
    \texttt{n + 1} & \texttt{ if n \texttt{mod} 2 = 0 }\\
    \texttt{n} & 
  \end{cases}\label{eq:cases} 
\end{align}
