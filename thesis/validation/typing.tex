\documentclass[11pt,oneside,a4paper]{report}

\begin{document}
\section{Types and validation}
The spellchecking equivalent for computer programs could be type checking; the problem of validating a programmers intuition of a program's intent.
Types also have other properties than simply validating they can in fact be related to thorems to which an implementation is the proof~\cite{howard1980formulae}.
\begin{lstlisting}[language=ML,caption={Head implementation},label={lst:headimpl}]
fun head l: List a -> a = 
    match l
        | Cons x _ -> x;
        | Nil -> ?;
    ;
\end{lstlisting}
For instance consider the implementation of type \texttt{List a -> a} in \autoref{lst:headimpl}, a total implementation of the function cannot exist.

The type system for the $L$ language will be the Hindley-Milner type system~\cite{hindley1969principal,milner1978theory}.

\subsection{The language of types}
Before delving into types the lambda calculus defined in \autoref{sec:lc} must be agumented with the \textit{let expression} (\autoref{eq:letb}).
\begin{align}
	\texttt{let } x = Y \texttt{ in } E
	\label{eq:letb}
\end{align}
It should be noted that the let binding can be expressed by abstraction and application (\autoref{eq:letaa}).
\begin{align}
	(\lambda x . E) (Y)
	\label{eq:letaa}
\end{align}
The let expression has a nice property that will become apparent later when typing rules are introduced.
The untyped lambda calculus exists at the ``value level'' while the now introduced syntax exists at the ``type level''.
There are two vairants of types in Hindley-Milner, the \textit{monotype} and the \textit{polytype}.
A monotype is either a variable, an abstraction of two monotypes or an application (\autoref{eq:mono}).
\begin{align}
	mono \,\,\tau = a \,|\, \tau \rightarrow \tau \,|\, C \tau_1 \dots \tau_n
	\label{eq:mono}
\end{align}
The application term of the monotype is dependant on the primitive types of the programming language.
In $L$ the set of monotypes are $\{ \texttt{Int}, \texttt{Bool}, \texttt{ADT} \}$.
The types $\tau_1 \dots \tau_n$ are the typevariables required to construct some atomic type.
An instance of which could be a \texttt{List a} with one type parameter \texttt{a}.
A polytype is a polymorphic type (\autoref{eq:poly}).
\begin{align}
	poly \,\, \sigma = \tau \,|\, \forall a . \sigma
	\label{eq:poly}
\end{align}
A central component of typing in Hindley-Milner is the \textit{environment}.
The environment list $\Gamma$ is a list of pairs of variable and type (\autoref{eq:env}).
$\Gamma \vdash x: \sigma$ implies a \textit{typing judgment} meaning that given $\Gamma$ the variable $x$ has type $\sigma$.
\begin{align}
	\Gamma \,\, = \epsilon \,|\, \Gamma, x : \sigma
	\label{eq:env}
\end{align}

Like in the untyped lambda calculus types also have notions of free and bound type varables.
Varables are bound when they have been introduced by a quantification or exist in the environment.
\begin{align}
	 & \textit{free}(a) = \{ a \}                                                              \\
	 & \textit{free}(C \tau_1 \dots \tau_n ) = \bigcup_{i-1}^n \textit{free}(\tau_i)           \\
	 & \textit{free}(\Gamma) = \bigcup_{x:\sigma \in \Gamma} \textit{free}(\sigma)             \\
	 & \textit{free}(\forall a . \sigma) = \textit{free}(\sigma) - \{ a \}                     \\
	 & \textit{free}(\Gamma \vdash x : \sigma) = \textit{free}(\sigma) - \textit{free}(\Gamma)
\end{align}

\subsection{Hindley-Milner rules}
With the now introduced primitives the Hindley-Milner type system is but a set of inference rules composed by said primitives.
\begin{figure}[ht]
	\begin{mdframed}
		\minipage{0.49\textwidth}
		\begin{prooftree}
			\AxiomC{$x: \sigma \in \Gamma$}
			\LeftLabel{Var}
			\UnaryInfC{$\Gamma\vdash x:\sigma$}
		\end{prooftree}
		\endminipage
		\minipage{0.49\textwidth}
		\begin{prooftree}
			\AxiomC{$\Gamma \vdash e_1 : \tau_1 \rightarrow \tau_2$}
			\LeftLabel{App}
			\AxiomC{$\Gamma \vdash e_2 : \tau_1$}
			\BinaryInfC{$\Gamma \vdash e_1 e_2 : \tau_2$}
		\end{prooftree}
		\endminipage\hfill\vspace{0.8cm}

		\minipage{0.49\textwidth}
		\begin{prooftree}
			\AxiomC{$\Gamma, x: \tau_1 \vdash e : \tau_2$}
			\LeftLabel{Abs}
			\UnaryInfC{$\Gamma \vdash \lambda x . e : \tau_1 \rightarrow \tau_2$}
		\end{prooftree}
		\endminipage\hfill
		\minipage{0.49\textwidth}
		\begin{prooftree}
			\AxiomC{$\Gamma \vdash e_1 : \sigma$}
			\LeftLabel{Let}
			\AxiomC{$\Gamma x : \sigma \vdash e_2 : \tau$}
			\BinaryInfC{$\Gamma \vdash \texttt{ let } x = e_1 \texttt{ in } e_2 : \tau$}
		\end{prooftree}
		\endminipage\hfill\vspace{0.8cm}

		\minipage{0.49\textwidth}
		\begin{prooftree}
			\AxiomC{$\Gamma \vdash e : \sigma_1$}
			\AxiomC{$\sigma_1 \sqsubseteq \sigma_2$}
			\LeftLabel{Ins}
			\BinaryInfC{$\Gamma \vdash e : \sigma_2$}
		\end{prooftree}
		\endminipage\hfill
		\minipage{0.49\textwidth}
		\begin{prooftree}
			\AxiomC{$\Gamma \vdash e : \sigma$}
			\AxiomC{$a \notin free(\Gamma)$}
			\LeftLabel{Gen}
			\BinaryInfC{$\Gamma e : \forall a . \sigma$}
		\end{prooftree}
		\endminipage
	\end{mdframed}
	\caption{Hindley-Milner type rules}
	\label{fig:hmrules}
\end{figure}
There are six rules in the Hindley-Milner rules outlined in \autoref{fig:hmrules}.
The first rule and also the only axiom is the Variable.
The Variable rule states that if some variable $x$ with type $\sigma$ has been deemed to exist, then they must be in the environment.
The Application rule states that if $e_1 e_2$ is of type $\tau_2$ then $e_1$ must conform to a type that can produce a type $\tau_2$ given a type $\tau_1$ and $e_2$ must conform to the type of this $\tau_1$.
The Instatiate rule are important to specify a quantified type $\sigma_1$ to a specific one $\sigma_2$.
Generalization lifts a type into a quantified type for all types which are bound.

When inferring types in Hindley-Milner it is important to keep track of which variables are bound in a generalization if some instantiation were to happen.

\subsection{Damas-Milner Algorithm W}
Typing rules are by themselves not that useful since they need all type information declared ahead of checking.
For this reason \textit{type inference} is considerably more useful.
Type inference is the techinque of automatically deriving types, of which there exist many algorithms.
One of the most common Hindley-Milner compatible algorithms is the Damas-Milner Algorithm W inference algorithm that infers correct types~\cite{damas1984type,damas1982principal}.
The Hindley-Milner rules will accept any rules inferred and accepted by Algorithm W.

The Damas-Milner Algorithm W rules (\autoref{fig:hmrules}) introduce some new concepts such as \textit{fresh variables}, \textit{most general unifier} and the \textit{substitution set}.
Fresh varables are introduced from a by picking a variable that has not been picked before from the infinite set $\tau_1, \tau_2 \dots $.
Fresh variables are introduced when unknown types are discovered and later unified.
Unification is performed differently based on the context.
In Hindley-Milner unification is between monotypes which can take one of three forms (\autoref{eq:mono}).
Note that the Var rules for most general unifier outlined in \autoref{fig:mgu} are commutative.
\begin{lemma}
	Var sub and Var empty are commutative.
\end{lemma}
\begin{proof}
	Var empty is trivially true since $\equiv$ is commutative and for any $a$ and $\tau_1$ the rule still produces $\emptyset$.\\\\
    The commutative property of Var sub comes from the realization that $S \cup \{ \tau_1 \rightarrow S\tau_2 \}$ and $S \cup \{ \tau_2 \rightarrow S\tau_1 \}$ lets Algorithm W accept on the same inputs.
    Furthermor note that Algorithm W substitutes and combines substitution sets at every step of the expression tree such that transitive types never occur because of the combination semantics.
    \begin{case}
        The types $\tau_1$ and $\tau_2$ are first introduced and used in unification.
        Either all future uses of $\tau_1$ will be mapped to $\tau_2$ by the substitution set or all future uses of $\tau_2$ will be mapped to $\tau_1$.
    \end{case}
    \begin{case}
        $\tau_1$ has been assigned in an earlier unification.
        If $\tau_2 \rightarrow \tau_1$ any existing references to $\tau_1$ need not change since all expressions of type variable $\tau_2$ will be mapped by the rules defined in \autoref{fig:hmrules}.
        If any future rules need the type variable $\tau_1$ and $\tau_1 \rightarrow \tau_2$ is introduced to the substitution set, the substitution rules in \autoref{fig:hmrules} will substitute the type.
    \end{case}
\end{proof}
\begin{figure}
	\begin{mdframed}
		\minipage{1\textwidth}
		\begin{prooftree}
			\AxiomC{$S,T \cup \{ (\tau_1, \gamma_1), (\tau_2, \gamma_2) \}$}
			\LeftLabel{Arrow}
			\UnaryInfC{$S, \{ (\tau_1 \rightarrow \tau_2, \gamma_1 \rightarrow \gamma_2) \} \cup T $}
		\end{prooftree}
		\endminipage\hfill\vspace{0.8cm}

		\minipage{1\textwidth}
		\begin{prooftree}
			\AxiomC{$S, T$}
			\AxiomC{$a \equiv \tau_1$}
			\LeftLabel{
				Var empty
			}
			\BinaryInfC{$S,\{ (a, \tau_1) \} \cup T $}
		\end{prooftree}
		\endminipage\hfill\vspace{0.8cm}

		\minipage{1\textwidth}
		\begin{prooftree}
			\AxiomC{$S \cup \{ a \rightarrow S\tau_1 \}, \{ a \rightarrow S\tau_1 \}T$}
			\AxiomC{$a \notin \textit{free}(\tau_1)$}
			\LeftLabel{Var sub}
			\BinaryInfC{$S, \{ (a, \tau_1) \} \cup T $}
		\end{prooftree}
		\endminipage\hfill\vspace{0.8cm}

		\minipage{1\textwidth}
		\begin{prooftree}
			\AxiomC{$S , \{ (\tau_1, \gamma_1) \dots , (\tau_n, \gamma_n)\}$}
			\AxiomC{$C_1 \equiv C_2$}
			\LeftLabel{Atom}
			\BinaryInfC{$S,C_1 \tau_1 \dots \tau_n,C_2 \gamma_1 \dots \gamma_n \cup T$}
		\end{prooftree}
		\endminipage\hfill\vspace{0.8cm}
	\end{mdframed}
	\caption{Rules for most general unification}
	\label{fig:mgu}
\end{figure}
\begin{figure}
	\begin{mdframed}[style=style1]
		\minipage{0.40\textwidth}
		\begin{prooftree}
			\AxiomC{$x: \sigma \in \Gamma$}
			\AxiomC{$\tau = \textit{inst}(\sigma)$}
			\LeftLabel{Var}
			\BinaryInfC{$\Gamma \vdash x:\tau , \emptyset$}
		\end{prooftree}
		\endminipage\hfill
		\minipage{0.64\textwidth}
		\begin{prooftree}
			\AxiomC{$\tau_1 = \textit{fresh}$}
			\AxiomC{$\Gamma, x: \tau_1 \vdash e: \tau_2, S$}
			\LeftLabel{Abs}
			\BinaryInfC{$\Gamma \vdash \lambda x . e : S\tau_1 \rightarrow \tau_2, S$}
		\end{prooftree}
		\endminipage\hfill\vspace{0.8cm}

		\minipage{1\textwidth}
		\begin{center}
			App
		\end{center}
		\vspace{-0.7cm}
		\begin{prooftree}
			\AxiomC{$\Gamma \vdash e_1 : \tau_1, S_1 \,\,\,\, \tau_3 = \textit{fresh}$}
			\AxiomC{$S_1 \Gamma \vdash e_2 : \tau_2, S_2 \,\,\,\, S_2 = \textit{mgu}(S_2 \tau_1, \tau_2 \rightarrow \tau_3)$}
			\BinaryInfC{$\Gamma \vdash e_1 e_2 : S_2 \tau_3, S_3 \cdot S_2 \cdot S_1$}
		\end{prooftree}
		\endminipage\hfill\vspace{0.8cm}

		\minipage{1\textwidth}
		\begin{prooftree}
			\AxiomC{$\Gamma \vdash e_1 : \tau_1, S_1$}
			\AxiomC{$S_1 \Gamma , x : S_1 \Gamma(\tau_1) \vdash e_2 : \tau_2, S_2$}
			\LeftLabel{Let}
			\BinaryInfC{$\Gamma \vdash \texttt{let } x = e_1 \texttt{ in } e_2 : \tau_2, S_1 \cdot S_2$}
		\end{prooftree}
		\endminipage
	\end{mdframed}
	\caption{Algorithm W rules}
	\label{fig:hmrules}
\end{figure}

\end{document}
