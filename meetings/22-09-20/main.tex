%! Author = valde
%! Date = 9/13/20

% Preamble
\documentclass[11pt]{article}

% Packages
\usepackage{amsmath}
\usepackage{minted}
\usepackage{hyperref}
\usepackage{xpatch,letltxmacro}

\hypersetup{
    colorlinks=true,
    linkcolor=blue,
    filecolor=magenta,
    urlcolor=cyan,
}
\urlstyle{same}

\LetLtxMacro{\cminted}{\minted}
\let\endcminted\endminted
\xpretocmd{\cminted}{\RecustomVerbatimEnvironment{Verbatim}{BVerbatim}{}}{}{}

\renewcommand{\figurename}{Listing}

% Document
\begin{document}
    \section{Emitting}
    C++ code, C++ lambda recursion.\\
    Inline all lambdas vs bind to value (untyped $\lambda$-calculus).
    \section{IR as $\lambda$-calculus}
    \href{http://caml.inria.fr/pub/papers/xleroy-zinc.pdf}{OCaml ZINC}
    \subsection{ADT}
    \textit{Mogensen–Scott encoding} Dana Scott.\\
    \textit{Church encoding}\\
    \textit{Boehm-Berarducci encoding}\\
    Function as data (or rather parameters).
    \begin{align}
        D = \{ c_i \}^N_{i = 1}, c_i \text{ has arity } A_i\\
    \end{align}
    $x_k$ is the constructor value for field $k \in \{1 \dots A_i\}$.\\
    $c_i$ is constructor $i \in N$, effectively.
    \[
        c_n: x_1 \rightarrow x_2 \dots \rightarrow x_{A_i} \rightarrow D
    \]
    Such that all constructors can be modelled as.
    \[
        \lambda x_1 \dots x_{A_i} . \lambda c_1 \dots c_N . c_i x_1 \dots x_{A_i}
    \]
    The "data" is extracted by: Given a set of functions $\{f_1 \dots f_N\}$ which each can handle union case $1 \dots N$.
    \begin{figure}[htp]
        \centering
        \begin{cminted}{haskell}
data Maybe a =
        | Nothing
        | Just a
        \end{cminted}
        \caption{Maybe in Haskell}
        \label{lst:maybe_in_haskell}
    \end{figure}\\
    Since we use untyped lambda calculus, we can model every parameter as \textit{any} type.
    Furthermore, the code in \autoref{lst:maybe_in_haskell} can be expressed in scott encoding.
    \begin{figure}[htp]
        \centering
        \begin{cminted}{haskell}
newtype MaybeAlgebra =
    MaybeAlgebra{ unMaybe :: forall a b. ((a -> b) -> b -> b) }

just :: a -> MaybeAlgebra
just a = \onjust onnothing -> onjust a

nothing :: MaybeAlgebra
nothing = \onjust onnothing -> onnothing

...
(just 5) (\x -> x) (12)
        \end{cminted}
        \caption{Maybe in Haskell as catamorphism}
        \label{lst:maybe_as_scott}
    \end{figure}





\end{document}